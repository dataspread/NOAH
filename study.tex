%!TEX root = main.tex

\section{Evaluation Study Design}
\label{sec:study}

In this section, we present the design of a user study
to evaluate whether 
\noah helps address spreadsheet navigational challenges.
\subsection{Study Design and Participants}
\label{sec:study_design}
\new{Our goal is to study the impact of an in-situ navigation 
plugin for spreadsheets, \noah, on navigation and exploration of data.
Therefore, we decided to compare a \noah-integrated 
spreadsheet system with a typical, popular one, Excel, across various tasks. Similar domain specific-evaluations have been performed for evaluating various  overview+detail interfaces, \eg database browsing~\cite{north2000snap} or 
tree navigation~\cite{beard1990navigational}. As explained in Section~\ref{sec:related}, the goals and user populations of spreadsheets and TDA tools are quite different. Therefore, we did not consider TDA tools for the comparative study.}
Our study was designed to answer the following questions:
\squishlist
\item {\bf RQ1.} %\sout{How does the addition of an overview help address spreadsheet navigational challenges, with respect to quantitative performance metrics such as speed and accuracy?}
\new{How does the integration of an overview plugin like \noah impact the efficiency of navigation
within and the usability of spreadsheet systems?}
\item {\bf RQ2.} %\sout{How do users make use of the aggregate column results in conjunction with the raw spreadsheet data to analyze data?} 
\new{How do the various components of \noah impact users' navigational experiences?}

%\item {\bf RQ3.} \sout{How does bin customization help users inject domain knowledge in addressing their specific exploration questions compared to the automated overview?} \sout{\saj{How do users utilize additional features of \noah, \eg bin customization, during navigation?}
%\item {\bf RQ4.} \sout{How is the user's subjective satisfaction with \noah compared to traditional spreadsheets?}}  \sout{\saj{How is the user's subjective satisfaction with \noah compared to the traditional spreadsheets when performing navigational tasks?}}
\squishend

\stitle{Study Design.} We conducted a $2 \times 2$ (2 datasets, 2 tools) mixed design within-subject study. The two tools used in the study were: Microsoft Excel, and \noah integrated within {\scshape DataSpread}~\cite{dataspread} \new{(henceforth, referred to as \noah for succinctness). As mentioned previously, we chose Excel for our comparative study because it is the most popular spreadsheet in use today.}
The study consisted of three phases: (a) an introductory phase explaining
the essential features of \noah via a video tutorial, followed by a warm-up session where participants explored a flight dataset~\cite{web:flight} in \noah 
to familiarize themselves with its features,
(b) a quiz phase where the participants first used both the tools
to perform targeted tasks on two different datasets (described later) 
followed by a survey to provide feedback on their impressions about Excel and \noah, and
(c) a semi-structured interview to collect qualitative feedback regarding the quiz phase. 

\stitle{Datasets.} We used two datasets---the birdstrikes (\new{used for evaluating visual data exploration tools like Keshif~\cite{yalccin2018keshif} and Voyager~\cite{wongsuphasawat2016voyager}}), and the Airbnb~\cite{web:airbnb} datasets. These datasets were chosen for their understandability to a general audience. The birdstrikes dataset records instances of birds hitting aeroplanes in different US states. The dataset has 10,868 records and 14 attributes (eight categorical, one spatial region, one temporal, four numeric). 
The Airbnb dataset was larger than the birdstrikes dataset. To ensure a fair comparison across tools, we created a sampled version of the original Airbnb dataset with 10,925 records, by uniformly sampling $10\%$ of the records from each US city. This dataset contained 15 attributes (six categorical, two spatial region, one temporal, six numeric).


\stitle{Participants.} We recruited 20 participants (11 female, 9 male) via flyers across the university and via a university email newsletter. 
The average age of the participants was $31.06$ years ($\sigma = 12.44$). 
The participants came from different backgrounds, 
\eg engineering (seven), business (five), administration (five), and natural science (three). 
During recruitment, 
prospective participants filled out an interest form
where they answered questions about their spreadsheet expertise\new{, their typical goals when using spreadsheets, and the spreadsheet operations they typically use}.
Participants were asked to rate their expertise with
different spreadsheet systems, \eg Excel and Google Sheets, 
and their frequency of using various spreadsheet 
tasks \eg data management, data analysis, statistical modeling, and what-if analysis.
We also asked participants about their familiarity 
with basic mathematical and statistical spreadsheet functions,
as well as advanced operations, \eg pivot table,
 \code{SUBTOTAL}, and conditional formatting. 
\new{To ensure that prior experience with spreadsheets 
didn't affect the performance of participants during the quiz phase}, 
we only recruited participants who rated their experience 
with Excel to be greater than four on a scale of one (no expertise at all) to five (very experienced). 
The selected participants were familiar with performing 
various tasks on spreadsheets, \eg maintaining, tracking, and
analyzing data, making predictions, and performing comparisons. 
All of the participants were familiar with the basic mathematical
and statistical functions supported by Excel. 
Each participant received \$10 per hour at the end of their session.

\subsection{Study Procedure}
\label{sec:procedure}
We now explain each of the phases of our study in more detail.
 
\stitle{Phase 1: Introduction to \noah.}
We began the study by showing a six-minute video tutorial explaining the features of \noah on a dataset of all the flights across the US for January 2018~\cite{web:flight}. The participants then explored the same dataset using \noah to familiarize themselves with the tool for about 10 minutes. The quiz phase began as soon as the participants finished their exploration. \toappendix{Note that we recruited only experienced Excel users for the study and therefore, we did not provide any introduction to Excel.}

\begin{table}[!htb]
\scriptsize
 \vspace{-10pt}
\caption{\new{Quiz tasks for the birdstrikes dataset. The task purposes and use cases correspond to the task typology discussed in Section~\ref{sec:usage}.}}
\label{tab:questions}
\centering
\new{
\begin{tabular}{l l}
\hline
Category & Question (Q), Purpose (P), Use case (U)    \\ \hline
steer                & \textbf{Q}: Organize the data by State. How many flights that had \\
                     & damages (damage = 1) originated from Florida?, \\
                     & \textbf{P}: \emph{Search}$\rightarrow$\emph{Query}, \textbf{U}: \code{lookup}$\rightarrow$\code{identify}\\ \hline
find                 & \textbf{Q}: How many flights in the currently visible spreadsheet \\
                     & window have damages?, \textbf{P}: \emph{Search}, \textbf{U}:  \code{browse}\\ \hline 
steer                & \textbf{Q}: Organize the data by State. How many flights that had \\
                     & damages (damage = 1) originated from California?, \\
                     & \textbf{P}: \emph{Search}$\rightarrow$\emph{Query}, \textbf{U}: \code{lookup}$\rightarrow$\code{identify}\\ \hline
\cmpA                & \textbf{Q}: Which state between Florida and California has a higher\\
                     & number of flights with damages?, \textbf{P}: \emph{Query}, \textbf{U}:  \code{compare}\\ \hline
\cmpB                & \textbf{Q}: Find the state with the most birdstrike occurrences,\\
                     & \textbf{P}: \emph{Query}$\rightarrow$\emph{Search}, \textbf{U}: \code{summarize}$\rightarrow$\code{locate}\\ \hline
customize            & \textbf{Q}: Organize the data by \emph{altitude}. What is the average cost of\\
                     & damages for altitude bin 0-450?, \textbf{P}: \emph{Query}$\rightarrow$\emph{Search}$\rightarrow$\emph{Produce}, \\
                     & \textbf{U}: \code{generate}$\rightarrow$\code{summarize}$\rightarrow$\code{lookup}\\
\end{tabular}
}
\end{table}


\stitle{Phase 2: The Quiz Phase.}
\new{The purpose of the quiz phase was to evaluate the effectiveness of \noah in addressing spreadsheet navigation limitations}. During the quiz phase, 
each participant performed specific tasks on the two datasets in two sessions, using Excel for one and \noah for the other. 
Each session was followed by a survey, described later. We
alternated the order of the datasets between consecutive participants. 
The order of the tools was alternated between every two participants. 
We developed an online JavaScript-based quiz system that recorded user responses and submission times. 
We also recorded the participants' interactions
with both tools using screen capture software. 
Participants were informed that they can refer to the Internet for help as many times as they wanted. However, due to their familiarity with Excel, none of the participants required external help. 
For reference, we also provided a printed handout to the participants 
that contained screenshots with the features of \noah. 

\emph{Quiz Tasks.} 
\new{We designed six tasks across five categories: 
steer (two tasks), find (one task), \cmpA (one task), \cmpB (one task), and customize (one task), encompassing six of the seven task typology use cases 
underlying the \emph{Search}, \emph{Query}, 
and \emph{Produce} purposes: \code{lookup}/\code{locate}, \code{identify},
\code{browse}, \code{compare}, \code{summarize}, and \code{generate} 
(see Table~\ref{tab:scope})\footnote{\new{We omitted the \code{export/save} use-case under the \emph{Produce} purpose since
\noah is not targeted at improving that use case. 
We also did not
study the \emph{Consume} purpose so as to focus our evaluation on completion 
of tasks as opposed to an open-ended exploration setting, which is beyond
the scope of our study.}}.
These tasks were selected to mimic a typical spreadsheet analysis
workflow and are representative of navigation interactions 
required for the most frequently issued spreadsheet operations~\cite{bradbard2014spreadsheet, lawson2009comparison}. The tasks were presented in the same order as shown in Table~\ref{tab:questions} for the birdstrikes dataset. 
%The order of the tasks mimics a spreadsheet navigation scenario where a user begins by casually exploring a dataset (\emph{Consume}), \eg organizing the birdstrike data by state, and then engaging in more focused set of tasks to satisfy various purposes, \eg \emph{Search} and \emph{Query} data, \emph{Produce} new information. 
The tasks for the Airbnb dataset mimic a scenario similar to the example in Section~\ref{sec:usage}. We explain the scenario in the context of the birdstrikes dataset next.}

\new{Say a user is interested in analyzing bird-strike statistics across US states. As the user is from Florida, she starts by computing bird-strike occurrences for that state (steer) and finds the number of occurrences to be quite high. After looking at the aggregate statistics, she decides to examine specific instances of bird-strike occurrences to inspect other attributes, \eg the altitude where the strike happened or the species of the bird (find). She notices a bird-strike occurrence at 50ft, which is surprising. She decides to investigate this issue later. For now, she focuses her attention on analyzing the state-wise statistics. She computes the same occurrence statistics for another large state, say California (steer) and then compares the statistics between the two large states (\cmpA). At this point, she becomes interested in learning the occurrence statistics across all states and in finding the state with the highest bird-strike occurrences (\cmpB). With the state-wise comparison completed, she decides to revisit low altitude bird-strikes. So she organizes the data by \emph{altitude} and computes the occurrence statistics at low altitudes, along with the associated damages (customize). For this final task, the bins generated by \noah did not correspond to the given altitude range and would require bin customization to compute the statistics}. 

%\hide{First,  the  user  is  interested  in  bird-strike  statistics  in  a state  of  interest,  say  Florida  (steer: lookup state  and  then identify the number of bird-strike occurrences). Then the user wants to find the actual bird-strike instances (find: \code{browse} data) within the spreadsheet region currently visible in the screen to obtain other details about the occurrence, for example, the altitude at which the strikes happened. Next, the user wants to verify whether the strike occurrences are similar for larger states (steer: \code{lookup} another state say California and then \code{identify} the number of bird-strike occurrences). Then the user \code{compare}s the occurrence statistics of the two states (\cmpA). Eventually the user decides to compare the same statistics for all the states in the dataset and find the state with the highest bird-strike occurrences (\cmpB: \code{summarize} all the states and then \code{locate} the maximum value state). Finally, the user wants to \code{summarize} the data by another attribute say altitude and \code{lookup} the bird-strike occurrence for a specific altitude range (customize). Note that in this scenario, the bins generated by \noah did not correspond to the given altitude range and would require the participant to utilize the bin customization feature to \code{generate} a customized overview first and then \code{summarize} and \code{lookup}.}


\emph{Survey.} After each session, 
participants rated the corresponding tool used on six metrics: 
confidence, comprehensibility, level of satisfaction, 
ease and speed of use, and ease of learning for spreadsheet navigation, 
on a Likert scale from one (\eg strongly disagree) to seven (\eg strongly agree). 
The survey asked multiple questions related to these metrics, 
15 in total, to ensure reliability. 
Participants were also asked to describe the positive 
and negative aspects of both tools. 
%\newsaj{The survey was designed to evaluate \emph{RQ1}}.

\emph{Evaluation.} 
We evaluated the accuracy and
completion time for each of the six tasks. 
\new{We combined this analysis with qualitative survey, interview, and screen/audio recording data to provide insights that can be corroborated across multiple sources. Moreover, we analyzed the survey responses to quantify the usability of both Excel and the \noah-integrated spreadsheet.} 
%For example, we analyzed the video recordings of participants' interaction with the tools during the quiz phase.  

\stitle{Phase 3: Interview Phase.} 
Following the survey, we conducted a semi-structured interview to 
identify participants’ preferred tools for different tasks 
and to understand the reasoning behind their choices. 
We also asked participants to comment on the usefulness of 
different features provided by \noah and Excel. 


