%!TEX root = main.tex


\section{\noah Use Cases}
\label{sec:usage}
\new{Users prefer spreadsheets 
over enterprise solutions 
to view, explore, and analyze data~\cite{chan1996use,raden2005shedding}. 
To understand the scope of typical user 
tasks on spreadsheets, 
we make use of the typology 
of abstract data exploration tasks~\cite{brehmer2013multi}---see 
Table~\ref{tab:scope}. 
This typology characterizes the range of domain-independent
tasks performed on visual representations of data,
developed after analyzing task classification 
systems in over two dozen papers, 
and has been applied to a variety of 
scenarios, including \newsaj{analytical reasoning during exploration~\cite{shrinivasan2008supporting}, multi-dimensional data analysis~\cite{valiati2006taxonomy}, graph comprehension~\cite{friel2001making}, cartography~\cite{roth2012empirically}, among others.}\agp{argue for
 utility with citations.} \agp{This doesn't cut it because i wanted to argue that it HAS been used, not that it can be used. So can we find citations post 2013?.}
While all the tasks in Table~\ref{tab:scope} 
can be performed using spreadsheets, 
\noah enhances the experience 
for many of these tasks, 
indicated by a checkmark (\checkmark). 
We describe these tasks in the context
of a real usage scenario for \noah below.
Additional details regarding how \noah 
complements spreadsheets 
in accomplishing these tasks can be found in 
the Appendix.}

% \begin{table}[!htb]\scriptsize
% \vspace{-10pt}
% \caption{Use cases of \noah, employing Brehmer and Munzner's typology~\cite{brehmer2013multi}.}
% \label{tab:scope}
% \vspace{-10pt}
% \centering
% \begin{tabular}{c p{2.8in}}
% \hline
% Purpose & Use Cases   \\ \hline
% \emph{Consume}         & \code{discover} (\checkmark), \code{present} (\checkmark), \code{enjoy} (\checkmark)\\
% \emph{Produce}         & \code{create} (\checkmark), \code{annotate} ($\times$)\\
% \emph{Search}            & \code{browse} (\checkmark), \code{explore(\checkmark)}, \code{locate} ($\times$), \code{lookup} ($\times$)\\
% \emph{Query}    &  \code{identify} (\checkmark), \code{summarize} (\checkmark), \code{compare} (\checkmark)  \\ \hline
% \end{tabular}
% \vspace{-5pt}
% \end{table}


\begin{table}[!htb]\scriptsize
\new{
\vspace{-10pt}
\caption{Example use cases where \noah provides benefits beyond spreadsheets (labeled by \checkmark if improved; $\times$ if it
remains the same), employing Brehmer and Munzner's typology~\cite{brehmer2013multi}.}
\label{tab:scope}
\vspace{-10pt}
\centering
\begin{tabular}{c p{2.8in}}
\hline
Purpose & Use Cases   \\ \hline
\emph{Consume}         & 
\code{discover} (\checkmark: {\em generation of hypotheses}, e.g., Rebecca finds a trend in larger cities and wants to check if it is present in smaller cities), 
\code{present} (\checkmark: {\em communication of information}, e.g., Rebecca sees the overall availability trends in the context of raw listings, and can present this view to her co-workers), 
\code{enjoy} (\checkmark: {\em casual encounters with visualization}, e.g., Rebecca uses the overview ``at a glance'' 
to understand which cities are present in the dataset, 
and how many listings are present per city)\\
\emph{Search}            & 
\code{explore/browse} (\checkmark: {\em searching based on characteristics where location is unknown/known}, e.g., Rebecca tries to find Chicago listings with availability greater than 60 days), 
\code{locate/lookup} (\checkmark: {\em searching based on entities where location is unknown/known}, e.g., Rebecca wants to find all entries corresponding to a given city like Chicago) \\
\emph{Query}    		&  
\code{identify} (\checkmark: {\em returning the characteristics of entity found during search}, e.g., Rebecca wants to examine Chicago listings to assess typical availabilities of listings in Chicago), 
\code{compare} (\checkmark: {\em returning characteristics of multiple entities}, e.g., Rebecca wants to compare listing patterns in Boston to that of Chicago), \code{summarize} (\checkmark: {\em returning characteristics of several entities}, e.g., Rebecca wants to gain an understanding of overall rental patterns across cities)  \\ 
\emph{Produce}         & \code{export/save} ($\times$), \code{generate/record} (\checkmark: {\em generation or recording of new information}, e.g., Rebecca issues an aggregate formula to generate summary availability statistics across cities)\\
\hline
\end{tabular}
}
\vspace{-5pt}
\end{table}


\new{We now describe a usage scenario 
that illustrates the benefits of integrating \noah
into typical spreadsheets. 
Let's assume that Rebecca, a journalist,
is exploring the {\em Inside Airbnb} dataset~\cite{web:airbnb},
a dataset of all the Airbnb listings across different US cities.}
This dataset was created to investigate the long-standing
accusation that many listings in Airbnb are illegally
run as hotel businesses, while avoiding taxes;
any listing available for rent for more than $60$ days
a year is considered to be operated as a hotel~\cite{accusation}. 

\new{Given that this is the first time she's examining
this dataset,
Rebecca wants to first gain a bird's eye view of the data. 
Without \noah, Rebecca 
would have had to use} 
a pivot table (discussed in Section~\ref{sec:related})
to construct
a summary---however, since this summary is disconnected
from the underlying data, it is hard for Rebecca to
map the summary statistics to
the raw data to obtain further details
about listings from any given city.
If she wanted to examine listings from a 
specific city,
Rebecca would have to either use search capabilities or
perform an explicit filter for this information,
and would have to switch back and forth between the
pivot table results and the raw listings,
present at disparate locations
on the spreadsheet.
\new{Even at the first step of exploration,
Rebecca would experience {\em substantial cognitive burdens,
loss of context,
and visual discontinuities},
with subsequent steps becoming progressively
more challenging.}



\new{Using \noah, 
she organizes the overview by city
and starts casually exploring the dataset, understanding
which cities are present, and roughly how many listings
does each city have---with \noah providing a high-level overview of
cities (Figure~\ref{fig:ux}a) (\code{enjoy}).}
The overview consists of sorted non-overlapping bins
containing one or more cities.
She can click on any bin and the corresponding
data will be displayed at the top of her screen.
\new{For example, clicking on the {\em Ashville-Boston} bin
displays the Ashville listings (Figure~\ref{fig:ux}c);
she can similarly find and examine properties of 
the Chicago listings by clicking on the {\em Chicago-Denver} bin (\code{locate} followed by \code{identify}).}
She can also zoom into bins using the ``\textcolor{blue}{$\rangle$}'' arrows,
zoom out of bins using the ``\textcolor{blue}{$\langle$}'' arrows,
and pan by clicking on various bins at the same level.
We discuss the construction of the overview
and associated interactions in Section~\ref{sec:ui}.


\new{Next, say Rebecca wants to analyze
one of the larger cities to understand the 
overall renting pattern (\code{summarize}).
She studies a few cities at a time, examining and comparing
the number of listings for each city, as displayed on the overview (\code{compare})}.
She decides to focus on Boston, her hometown,
and wants to find out how many listings
in Boston violate
the ``rent availability $>60$ days'' condition \new{(\code{identify})}.
In a typical spreadsheet, Rebecca needs to manually steer and then select
the Boston listings as input to a \code{COUNTIF} 
formula that counts the number of
rows that satisfy the above mentioned condition.
Using \noah, she can zoom into
the {\em Ashville-Boston} bin
(Figure~\ref{fig:concept}a and~\ref{fig:concept}b)
and then issues
a \code{COUNTIF} operation on the overview  \new{(\code{generate})}.
%\ie a \code{COUNTIF} formula\footnote{\code{COUNTIF} is a
%type of formula in spreadsheets that counts the number %of
%rows that satisfy some condition.}. 
The result is displayed as an {\em aggregate column}
alongside the overview (Figure~\ref{fig:ux}b).
Rebecca learns that 
more than half of the listings in Boston are 
effectively operating as hotels \new{(\code{discover})}---a large number! 

Based on this insight, Rebecca then wants to
understand availability statistics for an even
larger city, Chicago \new{(\code{compare})}.
As she uses the overview to navigate
to Chicago, \noah
automatically updates the aggregate column to
the \code{COUNTIF} formula results for Chicago \new{(\code{identify})},
without Rebecca needing to reissue it by performing a 
cumbersome steering operation as in traditional spreadsheets. 
Rebecca learns that Chicago exhibits 
a similar renting pattern as Boston, with more than half the listings operating as hotels. 
\new{She can then hypothesizes that this trend may hold for all large cities,
and can check whether the smaller cities have a different pattern (\code{discover}). Note that, the rows that satisfy the 
``rent availability $>60$ days'' condition, are listed
 in the spreadsheet adjacent
to the overview in sky blue (Figure~\ref{fig:ux}g) (\code{explore})}. 
With the raw 
data presented side-by-side, she can also dive into other attributes
of the listings operating as hotels to see if there are any other identifying characteristics,
\eg if they are all managed by a small number of agencies acting as individual renters \new{(\code{identify})}.

 Finally, as Rebecca navigates the data,
 her navigation history (Figure~\ref{fig:ux}d),
 \ie recently visited cities, and current navigation path
 (Figure~\ref{fig:ux}e) are kept up-to-date,
 allowing her to maintain context during navigation \new{(\code{record})}.
 She can revisit any previously visited cities \new{(\code{lookup})}
 by simply clicking on the relevant path
 in the navigation history.


Overall, with \noah, users can quickly
comprehend the data via the overview, access any region 
within the data without having to
scroll endlessly, and request additional details on demand without having to
steer across multiple screens. As users navigate and analyze the data, 
they can revisit previously accessed data via the navigation history, not losing context
of what they have explored. 





