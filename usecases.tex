%!TEX root = main.tex


\section{\noah Use Cases}
\label{sec:usage}
\saj{As explained in Section~\ref{sec:intro}, users prefer spreadsheets over enterprise solutions to view, directly manipulate, and explore data~\cite{chan1996use,raden2005shedding}. To identify the scope of these tasks on spreadsheets (see Table~\ref{tab:scope}), we draw parallel to the typology of abstract data exploration tasks~\cite{brehmer2013multi}. The typology characterizes the exploration and manipulation tasks performed on visual representations of data. While all the tasks in Table~\ref{tab:scope} can be performed using spreadsheets, \noah enhances the experience for a subset of these tasks, indicated by a checkmark (\checkmark). We explain how \noah complements spreadsheets in accomplishing these tasks in Appendix.}

\begin{table}[!htb]\scriptsize
\caption{Uses cases of \noah.}
\label{tab:scope}
\centering
\begin{tabular}{c c}
\hline
Purpose & Use Cases   \\ \hline
\emph{Consume}         & \code{discover} (\checkmark), \code{present} (\checkmark), \code{enjoy} (\checkmark)\\
\emph{Produce}         & \code{create} (\checkmark), \code{annotate} ($\times$)\\
\emph{Search}            & \code{browse} (\checkmark), \code{explore (\checkmark)}, \code{locate} ($\times$), \code{lookup} ($\times$)\\
\emph{Query}    &  \code{identify} (\checkmark), \code{summarize} (\checkmark), \code{compare} (\checkmark)  \\ \hline
\end{tabular}
\end{table}

\saj{We now describe a usage scenario that captures the spreadsheet exploration tasks accommodated by \noah (Table~\ref{tab:scope}) while illustrating the benefits of integrating an overview with spreadsheets. Let's assume that} Rebecca, an investigative journalist
is exploring the Inside Airbnb dataset~\cite{web:airbnb},
a dataset of all the Airbnb listings across different US cities.
This dataset was created to investigate the long-standing
accusation that many listings in Airbnb are illegally
run as hotel businesses, while avoiding taxes;
any listing available for rent for more than $60$ days
a year is considered to be operated as a hotel~\cite{accusation}. 

Without any prior knowledge of the dataset,
Rebecca starts by finding out information
about cities in the dataset and their corresponding listings;
for example, which cities are present,
and roughly how many listings does each city have \saj{(\code{browse} and \code{explore})}.
Without \noah, Rebecca may have used
a pivot table (discussed in Section~\ref{sec:related})
to construct
a summary---however, since this summary is disconnected
from the underlying data, it is hard for Rebecca to
map the summary statistics to
the raw data to obtain further details
about listings from any given city.
If she wanted to identify listings from a specific city \saj{(\code{lookup} and \code{locate} or \code{identify})},
Rebecca would have to either use search capabilities or
perform an explicit filter for this information,
and would have to switch back and forth between the
pivot table results and the raw listings,
present at disparate locations
on the spreadsheet.
Even at the first step of this exploration,
Rebecca would experience substantial cognitive burden,
loss of context,
and visual discontinuities,
with subsequent steps becoming progressively
more challenging.


Using \noah, she explores
the data by city,
with \noah providing a high-level overview of
cities (Figure~\ref{fig:ux}a).
The overview consists of sorted non-overlapping bins
containing one or more cities.
She can click on any bin and the corresponding
data will be displayed at the top of her screen.
For example, clicking on the {\em Ashville-Boston} bin
displays the Ashville listings (Figure~\ref{fig:ux}c).
She can also zoom into bins using the ``\textcolor{blue}{$\rangle$}'' arrows,
zoom out of bins using the ``\textcolor{blue}{$\langle$}'' arrows,
and pan by clicking on various bins at the same level.
We discuss the construction of the overview
and associated interactions in Section~\ref{sec:ui}.


Next, say Rebecca wants to analyze
one of the larger cities \saj{(\code{query}/\code{summarize}) to understand the renting pattern}.
She identifies these cities by
comparing the number of listings
for each city displayed on the overview \saj{(\code{compare})}.
She decides to focus on Boston, her hometown,
and wants to find out how many listings
in Boston violate
the ``rent availability $>60$ days'' condition.
In a typical spreadsheet, Rebecca needs to manually steer and then select
the Boston listings as input to a \code{COUNTIF} formula that counts the number of
rows that satisfy the above mentioned condition.
Using \noah, she can zoom into
the {\em Ashville-Boston} bin
(Figure~\ref{fig:concept}a and~\ref{fig:concept}b)
and then issues
a \code{COUNTIF} operation on the overview.
%\ie a \code{COUNTIF} formula\footnote{\code{COUNTIF} is a
%type of formula in spreadsheets that counts the number %of
%rows that satisfy some condition.}. 
The result is displayed as an {\em aggregate column}
alongside the overview (Figure~\ref{fig:ux}b).
Rebecca learns \saj{(\code{discover})} that more than half of the listings in Boston are 
effectively operating as hotels---a large number! 

Based on this insight, Rebecca then wants to
understand availability statistics for an even
larger city, Chicago \saj{(\code{summarize})}.
As she uses the overview to navigate
to Chicago, \noah
automatically updates the aggregate column to
the \code{COUNTIF} formula results for Chicago,
without Rebecca needing to reissue it by performing a cumbersome steering operation as in traditional spreadsheets. 
Rebecca learns that Chicago exhibits 
a similar renting pattern as Boston, with more than half the listings operating as hotels. She can then check if this trend holds for all large cities \saj{(\code{compare})},
and whether the smaller cities have a different pattern \saj{(\code{discover})}. \saj{Note that, the rows that satisfy the ``rent availability $>60$ days'' condition, are listed
 in the spreadsheet adjacent
to the overview in sky blue (Figure~\ref{fig:ux}g)}. With the raw 
data presented side-by-side, she can also dive into other attributes
of the listings operating as hotels to see if there are any other identifying characteristics,
\eg if they are all managed by a small number of agencies acting as individual renters \saj{(\code{identify})}.

 Finally, as Rebecca navigates the data,
 her navigation history (Figure~\ref{fig:ux}d),
 \ie recently visited cities, and current navigation path
 (Figure~\ref{fig:ux}e) are kept up-to-date,
 allowing her to maintain context during navigation.
 She can revisit any previously visited cities
 by simply clicking on the relevant path
 in the navigation history.


Overall, with \noah, users can quickly
comprehend the data via the overview, access any region 
within the data without having to
scroll endlessly, and request additional details on demand without having to
steer across multiple screens. As users navigate and analyze the data, 
they can revisit previously accessed data via the navigation history, not losing context
of what they have explored. 





