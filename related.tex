%!TEX root = main.tex

\section{Related Work}
\label{sec:related}
We now discuss tools and techniques
that partially address the limitations of navigating
spreadsheets.

\subsection{Spreadsheet Tools and Prototypes}
\cut{Commercial spreadsheet tools include
Microsoft Excel, Google Sheets, and Airtable---of
these, Excel and Google Sheets support features
targeted at improving users' navigation experience.}
\agp{Abrupt, add at least one line to say that existing spreadsheet tools recognize challenges in nav or something}
\newsaj{Existing spreadsheet tools---both commercial software and academic prototypes---recognize some of these navigation challenges while adopting different measures to address those.}

\cut{Excel allows users to maintain context during navigation
by manually creating named ranges~\cite{named-range},
that are essentially references
to a spreadsheet region,
and appear in a \emph{name box}
within the Excel menu bar.
Users can click on a named range
to navigate to the referred region.
However, the onus is on the user to create
named ranges for each region of interest.}
\stitle{Microsoft Excel.}
\new{Excel enables users to manually create references
to a spreadsheet region using the named 
ranges~\cite{named-range} feature,
accessible from the menu bar.
Users can click on a named range
to navigate to the referred region. 
However, the onus is on the user to create
named ranges for each region of interest. 
The pivot table~\cite{pivot} feature 
allows users to create a summary view to 
compare subsets of data without having to 
provides a summary view, enabling users
to compare subsets of data without
having to navigate to various locations
within the sheet.
This summary is placed
in a separate region of the spreadsheet,
preventing users from accessing
the data underlying the summary, impeding navigation. 
A similar overview feature, \code{SUBTOTAL}~\cite{subtotal},
adds a new row at the end of each distinct
subset of data with summary
information. Users can expand the summary to view
the actual spreadsheet data.
However, for datasets with many subsets (\eg for numeric data),
the number of new lines inserted (i.e., the summary) 
can itself become very
large, spanning multiple screens, and can cause
increased
visual discontinuity during navigation. Finally, NodeXL~\cite{hansen2010analyzing}
is a plug-in that provides a spreadsheet network overview
and supports navigational operations,
\eg zooming in/out, dynamic filtering, on the overview;
this plug-in only supports network datasets,
such as biological or social networks.}

\stitle{Google Sheets Explore.}
Google Sheets Explore~\cite{web:explore} provides
an overview of the data by
auto-generating charts of data statistics.
Users can specify queries to the system
(similar to a web search)
asking for different summary statistics.
While Explore
is a convenient means to understand
high-level data characteristics,
it doesn't address the
navigational challenges
related to scrolling and steering.  
 
%\subsection{Academic Spreadsheet Prototypes}
\cut{Existing academic work focuses
on adding a richer set of operations to spreadsheets,
or increasing their scalability.
We discuss work specifically targeting
navigation; other enhancements
such as managing hierarchical data~\cite{chang2016using}
or supporting joins~\cite{bakke2011spreadsheet},
are omitted.}

\stitle{Scalable Spreadsheet Summarization and Exploration.}
Smart-drill-down~\cite{joglekar2015smart}
generates an interactive summary of a large spreadsheet table as
a collection of rules; users can drill-down to
a specific rule to view more fine-grained rules.
Hillview~\cite{web:hillview} displays
the approximate results of group-by queries
on large spreadsheet tables.
While these tools support summarization at scale,
providing an overview of the spreadsheet,
they don't preserve spreadsheet semantics,
nor do they make it easy to scroll or steer
through large spreadsheets.
ABC~\cite{Raman99scalablespreadsheets} and {\sc DataSpread}~\cite{datamodels} support interactive
exploration of very large spreadsheet datasets,
beyond main-memory limits, maintaining spreadsheet look-and-feel,
but do not provide any new spreadsheet capabilities to assist
with navigation.
We build \noah as a plugin
to {\sc DataSpread}, since it is open-source.

\stitle{Interactive Tables.}
TableLens~\cite{rao1994table}
is a focus+context view
for browsing numerical information
in tables, looking
much like a spreadsheet with embedded bar charts.
Cells out of focus display
graphical bars proportional in length
to the underlying values,
providing a visual overview of the data,
while cells within the user's current focus
are magnified and display the graphical bars and
the raw data.
Ideas similar to TableLens
have been adopted by DataLens~\cite{bederson2004datelens}
for visualizing digital calendars, and
by FOCUS~\cite{spenke1996focus}
and InfoZoom~\cite{spenke2003visualization}
for exploring database query results.
Like TableLens, \noah
embeds graphical bars, but within
the overview to depict the underlying data
distribution. \noah captures
the user's current focus by highlighting
the corresponding bin in the overview.
While TableLens provides an easy mechanism
to get a high-level view of the data
and spot outliers, it
suffers from the same disadvantages
that focus+context views have relative
to overview+detail ones.
Unlike \noah, which supports
multiple granularities via binning,
TableLens only supports
one granularity (zoomed in or
zoomed out): beyond
a certain size, navigating (scrolling or steering)
the zoomed out data
is still cumbersome for users. Moreover, TableLens does not maintain the spreadsheet look-and-feel or capabilities.
%TableLens does not support non-numeric data,
%nor does it maintain the spreadsheet look-and-feel or capabilities
%(maintained in \noah due
%to the detailed view).


\stitle{Visual Interactive Spreadsheets.}
VisSh~\cite{nunez2000vissh}, SI~\cite{levoy1994spreadsheets},
SSR~\cite{chi1997spreadsheet}, ASP~\cite{piersol1986object},
and PhotoSpread~\cite{kandel2008photospread}
extend the input/output capabilities of cells
within spreadsheets, to
display charts, animation, photos, or geometric objects, or
accept input via direct manipulation dialogs,
among others.
While these tools allow users to represent and manipulate
data in a more flexible manner,
which in turn could help users getting a
high-level sense of the data,
they do not necessarily help users navigate data
more effectively.





\subsection{Spreadsheet Alternatives}
%Finally, we discuss solutions to navigational and summarization
%challenges in other domains.

\stitle{Overview+Detail Interfaces.}
Cockburn et al.~\cite{cockburn2009review}
provides a detailed survey of overview+detail
and zooming interfaces.
To improve navigation within large documents,
overview+detail interfaces~\cite{cockburn2006faster,kratz2010semi} allow users to interact with an overview as they explore the document. Zooming interfaces~\cite{nekrasovski2006evaluation,plaisant2002spacetree} provide a multi-granularity overview of the data and support interactions like zoom in/out to navigate across various granularities.
We follow the same analogy of
providing an overview of the spreadsheet first,
allowing users to drill-down further.

\stitle{Multiple Coordinated Views.} Multiple coordinated views~\cite{roberts2007state}, \eg \emph{Snap}~\cite{north2000snap}, \emph{Elastic Documents}~\cite{badam2018elastic} connect multiple views, for example, an overview and a detailed data view while enabling coordination between these views through brushing and linking. Similarly, \noah connects tabular spreadsheets with an overview and updates the state of the spreadsheet as users interact with the overview and vice-versa.

\stitle{Tabular Data Analysis (TDA).}
Visualization tools such as Tableu~\cite{tableau}, Power BI~\cite{powerbi}, Keshif~\cite{yalccin2018keshif}, Voyager~\cite{wongsuphasawat2016voyager} and
analytical tools such as SPSS~\cite{noruvsis1986spss}, SAS~\cite{sas1985sas}, can all provide summaries of tabular data in various forms (visualizations,
aggregate statistics). 
\new{These summaries 
are static overviews of the data---much like pivot tables, 
these summaries are not dynamically linked
to nor are co-located with
the underlying raw data. \newsaj{For example, Keshif~\cite{yalccin2018keshif} can display all the unique values corresponding to an attribute of interest, \eg cities of the Airbnb data~\cite{web:airbnb}. However, users cannot view or inspect the raw data corresponding to each city. With TDA tools, the spreadsheet look-and-feel is lost,
and as a result, users
loose the capability to directly manipulate raw data, derive new data, and issue formulae for free-form analysis.}  
\cut{For example, using these tools users cannot \code{browse} subsets of data and \code{lookup} or \code{locate} data points of interest (see Table~\ref{tab:scope}). Aditya: previous line problematic: they can... because all these tools support to varying extents the typology. Even users of such tools want to inspect the underlying data to verify the summary statistics and visualizations~\cite{2017trust}. Aditya:this line is problematic too... because the citation is to an AQP tool rather than some general purpose study}
\newsaj{Therefore, the goals of spreadsheets differ from TDA 
tools in two ways: a) facilitating direct manipulation of raw 
data and b) enabling arbitrary derivation of new data and summaries
using various operations involving navigation,
\eg formulas or expressions.}
\noah being a plug-in to spreadsheets, 
facilitates a unified interface that
upholds both these goals
while enhancing
navigational capabilities
for spreadsheet users.
}