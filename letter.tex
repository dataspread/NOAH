%!TEX root = main.tex


\section*{Resubmission of InfoVis 2019 submission ``Extending Spreadsheets to Support Seamless
Navigation at Scale'' to TVCG}

A previous version of this paper was submitted to InfoVis 2019. This version was rejected as the paper co-chairs deemed that revising the paper would require more time than allocated to meet reviewer expectations. The paper co-chairs mentioned the following: ``\textit{We particularly encourage such revisions where submissions were positively received by reviewers, but the revisions required were deemed to be beyond the scope of the conference review cycle. If you address the issues raised and subsequently submit to TVCG, please make reference to this  InfoVis submission and include a description of how you addressed the InfoVis reviewers' comments.}'' The entire set of reviews are attached. In particular, the reviewers asked for the following changes for a revision: 
\begin{itemize}
    \item[\textbf{C1:}] More clearly articulate the merits over alternative tools, such as Keshif. 
     \item[\textbf{C2:}] Convincingly explain the design choice around the binning mechanism or revise
    that approach.
    \item[\textbf{C3:}] Justify the choice of comparing with Excel and note the limitations of that
    approach.
     \item[\textbf{C4:}] Better justify the choice of tasks in the study. 
     \item[\textbf{C5:}] Explain the choice of the four research questions.
     \item[\textbf{C6:}] Provide more details about the study, including reporting intra-participant
    differences.
\end{itemize} 
We have spent the last four months 
preparing a revised version of this paper. We thank the reviewers and meta-reviewer for their detailed,
constructive feedback. 
We have taken the feedback very seriously in preparing
the revised version, and we believe the paper is substantially
stronger as a result. \newsaj{The revision necessitated a number of updates to the content of each section and are highlighted in teal color. In Section 1, we cited existing work that highlight the widespread adoption of spreadsheets even among users of advanced enterprise solutions, adding further merit to our goal of addressing navigation challenges within spreadsheets. Recent discourse within the visualization community also echoes our sentiment\footnote{https://twitter.com/FILWD/status/1187411664611749888}---instead of designing a new tool that may cater to a small population, we aim to improve user experience with spreadsheet tools boasting more than $700$ million users. Moreover, we augmented the definitions of spreadsheet interactions like scrolling and steering and clearly articulated the challenges of designing a general purpose plug-in for spreadsheets. In Section 2, we defined the scope of tasks supported by spreadsheets guided by existing taxonomy of abstract visualization tasks. We further enhanced our usage scenario by highlighting how different use cases of \noah maps to the scope defined for spreadsheets. In Section 5, we added additional discussions to clearly articulate the difference in the goals of spreadsheets spreadsheets and TDA tools which further justifies choice of Excel, the most widely used spreadsheet tool, as our baseline in the evaluation study. In Section 4, we added further clarification to the third design consideration (\textbf{DC3}). In Section 5, we provided justifications for developing a multi-granularity overview while explaining why we opted for histogram as an overview representation. In Section 6, we further clarified the goals of the evaluation study while rephrasing the research questions to better reflect those goals. We didn't alter our previous research questions, rather grouped a few research questions into a more high level and less ambiguous question. Moreover, we explained our choice of the quiz tasks and discussed three limitations of our study in detail. We revamped Section 7 completely to better reflect our research questions while adding more qualitative observations regarding the user experience with \noah. We also included an additional analysis on the intra-participant differences. In Section 8, we first added a summary of our takeaways from the evaluation study while discussing additional limitations of \noah and future enhancement opportunities. We now discuss how the updates made to all of these sections addressed the changes suggested by the reviewers.}

\begin{enumerate}
\leftmargin=25pt \rightmargin=0pt \labelsep=5pt \labelwidth=20pt \itemindent=0pt \listparindent=0pt \topsep=0pt plus 2pt minus 4pt \partopsep=0pt plus 1pt minus 1pt \parsep=0pt plus 1pt \itemsep=\parsep

\item \textit{Merits over other tools} (\textbf{C1}): \saj{In the paper, we have identified the major contribution of \noah as its \emph{design as a general purpose navigation plug-in} to any existing spreadsheet. As spreadsheets have a massive user-base, enhancing exploration and formula computation on large datasets while maintaining the spreadsheet look and feel as much as possible, has the potential to impact a lot of users. Most of these users employ spreadsheets as their primary data management and analysis tool while shunning enterprise solutions with more advanced features. Therefore, our goal is to develop a solution to improve navigation within spreadsheets.} 

\item \textit{Explanation of the binning mechanism} (\textbf{C2}): We have explained our choice of the binning mechanism in the context of the third design consideration (\emph{DC3}). This explanation can be found under the title ``Why a Multi-granularity Binned Overview'' in Section~\ref{sec:design}. \saj{In brief, we opted for binning to provide a clear and concise representation of the overall data distribution while minimizing user's back and forth movement across multiple screens during navigation.} However, we acknowledge limitation of this choice for binning categorical data and discuss possible solutions in Section~\ref{sec:discuss} in the context of our user study findings. 

\item \textit{Justification of Excel as a Baseline} (\textbf{C2}):
We have provided justification for using Excel over tabular data analysis (TDA) tools a) by highlighting the appeal of spreadsheets among users who shun more advanced enterprise solutions (see Section~\ref{sec:intro}), b) by identifying the scope of tasks supported by spreadsheets and TDA tools (see Section~\ref{sec:usage} and~\ref{sec:related}), c) by explaining that the goal of spreadsheets is to present the raw data as is, amenable to editing, formula computation, and comparison of derived data unlike TDA tools that hide the data while providing summarized statistics (see Section~\ref{sec:related}), and d) by highlighting the importance of addressing the shortcomings of a tool, the user base of which far exceeds that of TDA tools (see Section~\ref{sec:study_design}). 

\item \textit{Enhancing the user study design} (\textbf{C4, C5}):
We have added additional explanations regarding the goal of the study while better articulating the research questions, provided a better justification for the choice of the tasks for the study (see Section~\ref{sec:study_design}) and added a detailed discussion on the limitations of the study. \saj{Specifically, we merged a number of research questions that explored how specific features of \noah affected participants' navigation experience into one broad question which explores their overall experience with \noah and its components. Moreover, we have justified the choice of the quiz tasks by a) relating their goals with the scope of tasks supported by spreadsheets (see Section~\ref{sec:usage}), and b) explaining how these tasks can reveal how people may use \noah for spreadsheet navigation. Finally, we have identified two limitations of the study, resulting from the sample size and the study design, which we justify based on the scope of the study.}

\item \textit{Reporting Intra-participant Differences} (\textbf{C6}):
Finally, we have performed additional analysis of the study data to report intra-participant submission time differences while providing insights on the trends that emerged from analyzing the data. \saj{For the latter, we find that, despite a few exceptions, across all tasks, participants' task submission times were faster using \noah. However, we already obtained similar observations from Figure~\ref{fig:timeBox} in Section~\ref{sec:rq1}. Therefore, we have included the results of the intra-participant analysis in appendix while briefly summarizing its outcomes in Section~\ref{sec:rq1}.}
\end{enumerate}
