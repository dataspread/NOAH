%!TEX root = main.tex


\section*{Resubmission of InfoVis 2019 submission ``Extending Spreadsheets to Support Seamless
Navigation at Scale'' to TVCG}

A previous version of this paper was submitted to InfoVis 2019. This version was rejected as the paper co-chairs deemed that revising the paper would require more time than allocated to meet reviewer expectations. The paper co-chairs mentioned the following: ``\textit{We particularly encourage such revisions where submissions were positively received by reviewers, but the revisions required were deemed to be beyond the scope of the conference review cycle. If you address the issues raised and subsequently submit to TVCG, please make reference to this  InfoVis submission and include a description of how you addressed the InfoVis reviewers' comments.}'' The entire set of reviews are attached. In particular, the reviewers asked for the following changes for a revision: 
\begin{itemize}
    \item[\textbf{C1:}] More clearly articulate the merits over alternative tools, such as Keshif. 
     \item[\textbf{C2:}] Convincingly explain the design choice around the binning mechanism or revise
    that approach.
    \item[\textbf{C3:}] Justify the choice of comparing with Excel and note the limitations of that
    approach.
     \item[\textbf{C4:}] Better justify the choice of tasks in the study. 
     \item[\textbf{C5:}] Explain the choice of the four research questions.
     \item[\textbf{C6:}] Provide more details about the study, including reporting intra-participant
    differences.
\end{itemize} 
We thank the reviewers and meta-reviewer for their detailed,
constructive feedback. 
Taking this feedback into account, 
we have spent the last four months 
preparing a revised version of this paper.
We believe the paper is substantially
stronger as a result. 
Every section in the paper has been revised;
the changes are highlighted in \new{blue}.
We briefly describe the changes to each section
before returning to the changes required in the revision 
({\bf C1}--{\bf C6}).
\subsection*{Changes Organized by Section}
\begin{itemize}
	\leftmargin=25pt \rightmargin=0pt \labelsep=5pt \labelwidth=20pt \itemindent=0pt \listparindent=0pt \topsep=0pt plus 2pt minus 4pt \partopsep=0pt plus 1pt minus 1pt \parsep=0pt plus 1pt \itemsep=\parsep
\item 
In Section~\ref{sec:intro}, we now cite
work that highlights the widespread adoption
of spreadsheets even among users of 
advanced enterprise solutions, further justifying 
why we focused on addressing navigation challenges within
spreadsheets. In fact, recent 
debate within the visualization community following
VIS 2019 also echoes this view\footnote{https://twitter.com/FILWD/status/1187411664611749888}---instead of designing a new sophisticated tool from scratch that 
may cater to a small population, 
we aim to enhance the user experience for
existing spreadsheet tools
with nearly a billion users.
We further augment   
the definitions of spreadsheet 
interactions like scrolling and steering in Section~\ref{sec:intro}, 
and now clearly articulate the challenges of 
designing a general-purpose plug-in for spreadsheets.
\item 
In Section~\ref{sec:usage}, 
our usage scenario has been substantially revised 
using Brehmer and Munzner's typology~\cite{brehmer2013multi} to
clarify the scope of tasks supported by spreadsheets, as well
as those that are enhanced
when using a spreadsheet with a \noah plug-in; our new Table~\ref{tab:scope}
provides use-cases supported within \noah
for each category.
\item
In Section~\ref{sec:related},
we now clearly
articulate the differences in goals
of spreadsheets and tabular data analysis tools (TDA),
further justifying our choice of Excel,
the most widely used spreadsheet tool,
as our baseline in the evaluation study.
\item 
In Section~\ref{sec:design},
we now explain the  
third design consideration 
for \noah more clearly
(\textbf{DC3})---focusing on motivating the binning
mechanism.
\item In Section~\ref{sec:ui}, 
we now justify why we are developing a 
multi-granularity overview,
and explain why 
we opted for histograms as an overview representation, \new{by contextualizing our
approach using prior work on multi-scale aggregation~\cite{elmqvist2009hierarchical}.} 
\item 
In Section~\ref{sec:study}, 
we further clarified the goals of our study
and reframed our research questions to 
better reflect those goals. 
Our research questions haven't been substantially altered;
rather, they have been grouped together and made more precise. 
In this version,
we now explain our choice
for the quiz tasks and discuss 
three limitations of our study in detail. 
\item Section~\ref{sec:results} has been completely
revamped to better reflect our research questions,
while adding more qualitative observations 
regarding the user experience with \noah. 
We now
include additional analysis results on intra-participant differences. 
\item In Section~\ref{sec:discuss}, we have added a summary of our takeaways from the evaluation study, while discussing additional limitations of \noah and future enhancement opportunities. 
\end{itemize}
\subsection*{Changes Organized by Reviewer Concerns (C1--C6)}


\begin{enumerate}
\leftmargin=25pt \rightmargin=0pt \labelsep=5pt \labelwidth=20pt \itemindent=0pt \listparindent=0pt \topsep=0pt plus 2pt minus 4pt \partopsep=0pt plus 1pt minus 1pt \parsep=0pt plus 1pt \itemsep=\parsep

\item \textit{Merits over other tools} (\textbf{C1}): 
Throughout the paper (and especially in 
Sections~\ref{sec:intro}, \ref{sec:related},
and \ref{sec:study}), we have emphasized
that the main contribution of \noah
is its \emph{design as a general-purpose navigation plug-in} 
to any existing spreadsheet system. 
As spreadsheets have a massive user-base, 
enhancing exploration and formula computation 
on large datasets while maintaining 
the spreadsheet look-and-feel as much as possible, 
has the potential to impact hundreds of millions 
of spreadsheet users. 
Most of these users employ spreadsheets 
as their primary data management and analysis 
tool while shunning enterprise solutions 
with more advanced features. 
Therefore, our goal is to develop a 
solution to improve navigation within spreadsheets.
Indeed, we could have tried to enhance navigation
in other tools, such as Keshif, or Tableau, like
the reviewers suggest---but we would be forcing
spreadsheet users to adopt an entirely new tool, something
that they are clearly loath to do. 

\item \textit{Explanation of the binning mechanism} (\textbf{C2}): 
We now explain our choice
of the binning mechanism in 
the context of the third design consideration (\emph{DC3}), 
as detailed in Section~\ref{sec:design}.
We further expand on this in the blurb titled 
``Why a Multi-granularity Binned Overview'' 
in Section~\ref{sec:ui}.
In brief, we opted for binning to 
provide a clear and concise 
representation of the overall data 
distribution while minimizing user's back 
and forth movement across multiple screens 
during navigation. 
At some level, we do need to limit the number
of values displayed so that the overview
fits on the screen: binning is a natural solution
for that issue. 
We cite related work on 
multi-scale aggregation~\cite{elmqvist2009hierarchical} 
to motivate the choice of 
a multi-granularity overview
and show that binned aggregation via histograms~\cite{liu2013immens}
is a suitable representation of such an overview.
However, we acknowledge limitations 
for this choice for binning categorical data 
and discuss possible solutions 
in Section~\ref{sec:discuss} 
in the context of our user study findings. 


\item \textit{Justification of Excel as a Baseline} (\textbf{C3}):
Since our goal is to improve spreadsheet
user experience---and there are very good reasons for doing so,
as outlined in item 1) above---the natural comparison point
is Excel. 
That said, beyond popularity, 
we have provided a thorough justification
for what spreadsheet systems like Excel offer
relative to tabular data analysis tools (TDA),
making it a more appropriate point of comparison.
Our justification for using Excel over 
tabular data analysis (TDA) tools involves 
a) highlighting the appeal of spreadsheets 
among users who shun more advanced enterprise 
tools (Section~\ref{sec:intro}), 
b) identifying the scope of tasks 
supported by spreadsheets and 
TDA tools (Section~\ref{sec:usage} and~\ref{sec:related}), 
c) explaining that the goal of spreadsheets 
is to present the raw data as is, 
amenable to editing, formula computation, 
and comparison of derived data, all done in-situ,
unlike TDA tools that hide the data 
while providing summarized statistics 
(Section~\ref{sec:related}), and 
d) highlighting the importance of 
addressing the shortcomings of a tool, 
the user base of which far exceeds 
that of TDA tools (Section~\ref{sec:study}). 


\item \textit{Enhancing the user study design} 
(\textbf{C4, C5}):
We have added additional 
explanations regarding the goals 
of the study while better articulating 
the research questions, 
provided a better justification 
for the choice of the tasks for the study (see Section~\ref{sec:study_design}) 
and added a detailed discussion on the limitations of the study. 
Specifically, we merged our previous \emph{RQ1} (evaluating the quiz task performance) 
with \emph{RQ4} (summarizing user feedback on performance via a survey) 
into a unified research question (new \emph{RQ1}) 
on the participants' navigation performance. 
Moreover, we merged previous \emph{RQ2} and \emph{RQ3} 
that explored how specific features of \noah 
affected participants' navigation experience 
into one broad question (new \emph{RQ2}) 
that explores their overall experience with \noah and its components. 
Moreover, we have justified the choice of the quiz tasks by 
a) relating their goals with the scope of tasks supported by spreadsheets (see Section~\ref{sec:usage}), 
and b) explaining how these tasks can reveal how people may use \noah 
for spreadsheet navigation. 
Finally, we have identified three limitations of the study, 
resulting from the sample size and the study design, which we justify based on the scope of the study.

\item \textit{Reporting Intra-participant Differences} (\textbf{C6}):
\newsaj{We performed additional analysis of the study data to report intra-participant submission time differences for the quiz phase tasks. While we summarize the results of the analysis in Section~\ref{sec:time_acc}, the detailed result can be found in Appendix}\agp{pls add; insights on trends sounds vague let's be more clear}. \newsaj{We find that, despite a few exceptions, across all tasks, participants' task submission times were faster using \noah---all but one participants completed at least four tasks in less time using \noah compared to Excel.} However, we already obtained similar observations from Figure~\ref{fig:timeBox} in Section~\ref{sec:rq1}. Therefore, we have included the results of the intra-participant analysis in Appendix.
\end{enumerate}
