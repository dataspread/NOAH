%!TEX root = main.tex


\section*{Resubmission of InfoVis 2019 submission ``Extending Spreadsheets to Support Seamless
Navigation at Scale'' to TVCG}

A previous version of this paper was submitted to InfoVis 2019. This version was rejected as the paper co-chairs deemed that revising the paper would require more time than allocated to meet reviewer expectations. The paper co-chairs mentioned the following: ``\textit{We particularly encourage such revisions where submissions were positively received by reviewers, but the revisions required were deemed to be beyond the scope of the conference review cycle. If you address the issues raised and subsequently submit to TVCG, please make reference to this  InfoVis submission and include a description of how you addressed the InfoVis reviewers' comments.}'' 

The present version of the paper addresses all of the issues mentioned by the reviewers. The entire set of reviews are attached. In particular, the reviewers asked for the following changes for a revision: 
\begin{itemize}
    \item More clearly articulate the merits over alternative tools, such as Keshif. 
     \item Convincingly explain the design choice around the binning mechanism or revise
    that approach.
    \item Justify the choice of comparing with Excel and note the limitations of that
    approach.
     \item Better justify the choice of tasks in the study. 
     \item Explain the choice of the four research questions.
     \item Provide more details about the study, including reporting intra-participant
    differences.
\end{itemize} 
We have spent the last four months 
preparing a revised version of this paper.
\agp{One thing that is confusing is that the bulletpoints above
don't neatly map to the bulletpoints below, or the enumerated list after that. Stopped reading after this}
Here, we provide a response to the reviews. 
In our revised submission,
(i) we discuss the merits of having an interface like \noah
integrated within an existing spreadsheet, as opposed
to other general-purpose exploratory tools such as  Keshif;
(ii) we justify our design choice surrounding the binning mechanism;
and 
(iii) we enrich the user study design section further 
by providing justifications for comparing \noah and Excel, 
rephrasing our research questions reflecting the purpose of the user study, explaining the choice of quiz tasks,
(iv) we reorganize the results section to  highlight key contributions of \noah while explaining participant experiences in using the tool. Moreover, we report the  intra-participant differences for the study. 
The major changes to the paper include:

\begin{enumerate}
\leftmargin=25pt \rightmargin=0pt \labelsep=5pt \labelwidth=20pt \itemindent=0pt \listparindent=0pt \topsep=0pt plus 2pt minus 4pt \partopsep=0pt plus 1pt minus 1pt \parsep=0pt plus 1pt \itemsep=\parsep

\item \textit{Explanation of the binning mechanism.} We have explained our choice of the binning mechanism in the context the third design consideration (\emph{DC3}) and can be found under the title ``Why a Multi-granularity Binned Overview'' in Section~\ref{sec:design}. \saj{In brief, we opted for binning to provide a clear and concise representation of the overall data distribution while minimizing user's back and forth movement across multiple screens during navigation.} However, we acknowledge the limitation of this choice for binning categorical data and discuss possible solutions in Section~\ref{sec:discuss}. 

\item \textit{Justification of Excel as a Baseline.}
We have provided justification for using Excel over tabular data analysis (TDA) tools a) by highlighting the appeal of spreadsheets in interacting with data even for the users of enterprise solutions (see Section~\ref{sec:intro}), b) by identifying the scope of tasks supported by spreadsheets and the TDA tools (see Section~\ref{sec:usage} and~\ref{sec:related}), c) by explaining that the goal of spreadsheets is to facilitate direct manipulation and arbitrary derivation of the data unlike TDA tools that hide the data while providing summarized statistics (see Section~\ref{sec:related}), and d) by highlighting the importance of addressing the shortcomings of a tool, the user base of which, far exceeds that of the TDA tools (see Section~\ref{sec:study_design}). 

\item \textit{Merits over other tools.} \saj{In the paper, we have identified the major contribution of \noah as its \emph{design as a general purpose navigation interface}, the concepts of which can be implemented and integrated as a plug-in to any spreadsheet. As spreadsheets have a massive user-base, enhancing the experience surrounding data exploration and formula computation while maintaining the spreadsheet look and feel as much as possible, has the potential to impact a lot of users. As we mentioned earlier, even users of enterprise solutions prefer Excel for direct data manipulation. Therefore, developing a solution to improve the experience with basic operations of such a tool has its merits.} 

\item \textit{Enhancing the user study design.}
We have added additional explanations on the goal of the study while better articulating the research questions, provided a better justification for the choice of the tasks for the study (see Section~\ref{sec:study_design}) and added a detailed discussion on the limitations of the study. \saj{Specifically, we merged a number of research questions that explored how specific features of \noah affected participants' navigation experience into one broad question which explores their overall experience with \noah and its components. Moreover, we have justified the choice of the quiz tasks by a) relating their goals with the scope of tasks supported by spreadsheets (see Section~\ref{sec:usage}, and b) explaining how these tasks can reveal how people may utilize \noah for spreadsheet navigation. Finally, we have identified two limitations of the study , resulting from sample size and the study design, which we justify based on the scope of the study.}

\item \textit{Reporting Intra-participant Differences.}
Finally, we have performed additional analysis of the study data to report intra-participant submission time differences while providing insights on the trends that emerged from analyzing the data. \saj{For the latter, we find that, despite a few exceptions, across all tasks, participants' task submission times were faster using \noah. However, we already obtained similar observations from Figure~\ref{fig:timeBox} in Section~\ref{sec:rq1}. Therefore, we have included the results of the intra-participant analysis in appendix while briefly summarizing its outcomes in Section~\ref{sec:rq1}.}
\end{enumerate}

\saj{In addition to the aforementioned changes, we have reorganized Section~\ref{sec:results} to reflect the rephrased research questions while supporting our findings from the study by providing qualitative observations, such as participants' comments and interaction patterns with the \noah. We have explained how different features of \noah affected participants' spreadsheet navigation experience, either positively or negatively. Based on these observations, we have updated Section~\ref{sec:discuss} to include additional limitations of \noah while highlighting future research opportunities.}